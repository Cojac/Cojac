% ==============
% Create the glossary
% ==============

\makeglossaries

\newglossaryentry{JDK}
{
    name = {Java Development Kit (JDK)},
    text = {JDK},
    description = {Un ensemble d'outils utilisés pour développer une application Java. Cela inclus la JVM, le compilateur, etc.}
}

\newglossaryentry{JVM}
{
    name = {Java virtual machine (JVM)},
    text = {JVM},
    description = {Une machine virtuelle qui permet d'exécuter des programmes écrits en Java ainsi qu'avec certains autres langages. Les programmes exécutés doivent préalablement être compilés en Bytecode.}
}

\newglossaryentry{Unums}
{
    name = {Universal number (Unum)},
    text = {Unum},
    description = {Un nouveau format de stockage pour les nombres réels dont le but est de remplacer les nombres à virgule flottante.}
}

\newglossaryentry{JNI}
{
    name = {Java Native Interface (JNI)},
    text = {JNI},
    description = {JNI permet d'appeler des fonctions natives (généralement écrites en C) depuis un programme Java.}
}

\newglossaryentry{COJAC}
{
    name = {COJAC},
    description = {Un outil capable de modifier le comportement d'une application Java à l'aide d'un agent Java.}
}

\newglossaryentry{Maven}
{
    name = {Maven},
    description = {Un outil d'automatisation pour gérer les dépendances et compiler une application.}
}

\newglossaryentry{Java-agent}
{
    name = {agent Java},
    plural = {agents Java},
    description = {Un objet Java qui peut intercepter le chargement d'une classe et modifier son Bytecode.}
}

\newglossaryentry{Bytecode}
{
    name = {Bytecode},
    description = {Le langage utilisée par la JVM}
}

\newglossaryentry{Behaviour}
{
    name = {Behaviour},
    description = {Un mécanisme de COJAC permettant de modifier les opérations agissant sur les types primitifs.}
}

\newglossaryentry{Wrapper}
{
    name = {Wrapper},
    description = {Un mécanisme de COJAC permettant de remplacer les \textit{floats} et \textit{doubles} par un objet.}
}

\newglossaryentry{Complex-number}
{
    name = {nombre complexe},
    plural = {nombres complexes},
    description = {Une extension des nombres réels permettant de faire certaines opérations mathématiques impossibles avec les nombres réels.}
}

\newglossaryentry{Posit}
{
    name = {Posit},
    description = {Un nouveau format de stockage spécifié dans les Unums III.}
}

\newglossaryentry{SoftPosit}
{
    name = {SoftPosit},
    text = {\textit{SoftPosit}},
    description = {Une librairie C permettant de réaliser certains calculs avec des Unums.}
}

\newglossaryentry{JAR}
{
    name = {Java ARchive (JAR)},
    text = {JAR},
    description = {Un fichier regroupant toutes les ressources et toutes les classes utilisées par une application Java.}
}

\newglossaryentry{Makefile}
{
    name = {Makefile},
    text = {\textit{Makefile}},
    description = {Un fichier de configuration composé de règles permettant de compiler un programme.}
}