\section{Conception}

Ce chapitre contient la conception de l'intégration des unums dans COJAC.

\subsection{Librairie}

Plusieurs librairies existantes permettent de gérer des unums. Cependant, aucune librairie n'a été trouvée pour Java. La librairie, qui est de loin la plus complète et documentée, est une librairie C++ nommée \textbf{universal} \cite{universal-github}.

\subsection{Approche}

Deux approches principales sont disponibles:
\begin{itemize}
    \item Créer une passerelle vers la librairie universal native \cite{universal-github}.
    \item Implémenter les unums en Java.
\end{itemize}

Ces approches sont très différentes et possèdent chacune des avantages et inconvénients différents.

\subsubsection{Passerelle vers une librairie native}

Avantages:
\begin{itemize}
    \item La librairie est complète et testée.
\end{itemize}

Inconvénients:
\begin{itemize}
    \item Cette option ne fonctionnera que sur certaines plateformes.
    \item Il faut également écrire du code natif pour faire la passerelle.
    \item Il est aussi nécessaire de libérer la mémoire lorsqu'il n'y a plus besoin de la classe.
    \item Des problèmes peuvent survenir en cas de concurrence.
\end{itemize}

\subsubsection{Implémenter les unums en Java}

Avantages:
\begin{itemize}
    \item Le support de toutes les plateformes est gardée.
\end{itemize}

Inconvénients:
\begin{itemize}
    \item La librairie doit être entièrement implémentée, mais la librairie native peut servir de modèle.
    \item La librairie doit être maintenue. Il est possible de créer un dépôt public séparé afin de laisser d'autres personnes y contribuer.
\end{itemize}

\subsubsection{Choix}

La passerelle vers la librairie native a deux défauts importants:
\begin{itemize}
    \item Il n'est pas possible de supporter toutes les plateformes. Cependant, le plugin NAR \cite{nar-maven-plugin} peut aider à supporter plus de plateformes.
    \item La librairie \textit{universal} fonctionne avec une classe C++. Ainsi, il est nécessaire de créer un objet et de garder cette instance. Cependant, la libération de la mémoire est un problème parce que la méthode \textit{finalize} qui pourrait être utilisée est dépréciée depuis Java 9 \cite{java-finalize-documentation}. Il est aussi possible de recréer une classe à chaque fois, mais cela réduira largement les performances.
\end{itemize}

Ainsi, cette fonctionnalité sera implémentée en Java dans une classe dédiée afin de pouvoir l'extraire dans un nouveau dépôt dans le futur. L'implémentation se basera sur l'implémentation de la librairie \textit{universal} qui possède une license MIT et sur les spécifications des unums.