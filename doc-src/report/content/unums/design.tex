\section{Conception}

Ce chapitre contient la conception de l'intégration des unums dans COJAC.

\subsection{Librairie}

Plusieurs librairies existantes permettent de gérer des unums. Cependant, aucune librairie n'a été trouvée pour Java.

Deux librairies avancées différentes ont été trouvées:
\begin{itemize}
    \item \textbf{Universal} \cite{universal-github}, une librairie C++
    \item \textbf{SoftPosit} \cite{softposit-github}, une librairie C
\end{itemize}

\subsection{Approche}

Deux approches principales sont disponibles:
\begin{itemize}
    \item Créer une passerelle vers une librairie native.
    \item Implémenter les unums en Java.
\end{itemize}

Ces approches sont très différentes et possèdent chacune des avantages et inconvénients différents.

\subsubsection{Passerelle vers la librairie native \textit{universal}}

Avantages:
\begin{itemize}
    \item La librairie est complète et testée.
\end{itemize}

Inconvénients:
\begin{itemize}
    \item Cette option ne fonctionnera que sur certaines plateformes.
    \item Il faut également écrire du code natif pour faire la passerelle.
    \item Il est aussi nécessaire de libérer la mémoire lorsqu'il n'y a plus besoin de la classe.
    \item Des problèmes peuvent survenir en cas de concurrence.
\end{itemize}

\subsubsection{Passerelle vers la librairie native \textit{SoftPosit}}

Avantages:
\begin{itemize}
    \item Les posits 8 et 16 bits ont beaucoup de tests. Le posit 32 bits est aussi testé, mais il peut encore y avoir des problèmes.
    \item Pas besoin d'alouer et libérer de la mémoire
\end{itemize}

Inconvénients:
\begin{itemize}
    \item Cette option ne fonctionnera que sur certaines plateformes.
    \item Il faut également écrire du code natif pour faire la passerelle.
    \item La librairie ne contient pas de posit 64 bits.
\end{itemize}

\subsubsection{Implémenter les unums en Java}

Avantages:
\begin{itemize}
    \item Le support de toutes les plateformes est gardée.
\end{itemize}

Inconvénients:
\begin{itemize}
    \item La librairie doit être entièrement implémentée, mais la librairie native peut servir de modèle.
    \item La librairie doit être maintenue. Il est possible de créer un dépôt public séparé afin de laisser d'autres personnes y contribuer.
\end{itemize}

\subsubsection{Choix}

La passerelle vers la librairie native a deux défauts importants:
\begin{itemize}
    \item Il n'est pas possible de supporter toutes les plateformes. Cependant, le plugin NAR \cite{nar-maven-plugin} peut aider à supporter plus de plateformes.
    \item La librairie \textit{universal} fonctionne avec une classe C++. Ainsi, il est nécessaire de créer un objet et de garder cette instance. Cependant, la libération de la mémoire est un problème parce que la méthode \textit{finalize} qui pourrait être utilisée est dépréciée depuis Java 9 \cite{java-finalize-documentation}. Il est aussi possible de recréer une classe à chaque fois, mais cela réduira largement les performances.
    \item La librairie \textit{SoftPosit}
\end{itemize}

Bien que la librairie \textit{SoftPosit} ne contienne pas de posit 64 bits, elle sera tout de même utilisé pour plusieurs raisons:
\begin{itemize}
    \item L'implémentation a été testée et les bugs devraient être rares.
    \item Une meilleure implémentation avec ses tests nécessiterait plus de temps qu'il reste d'ici la fin du projet.
    \item Si une implémentation 64 bits est ajoutée dans la librairie, elle pourra facilement être intégrée à COJAC.
    \item En cas de bugs ou de modifications de la spécification, il est assez probable que d'autres personnes mettent à jour la librairie
    \item La précision des posits 32 bits pourront tout de même être comparés avec les \textit{floats}.
\end{itemize}

\subsection{Passerelle vers le code natif}

Plusieurs alternatives existent pour faire la passerelle depuis le code Java vers le code natif en C:
\begin{itemize}
    \item JNI
    \item JNA
    \item SWIG
\end{itemize}

