\section{Format de stockage des nombres réels}

Il y a deux types principaux de nombres à utiliser: les nombres entiers et les nombres réels. Les nombres entiers sont simplement stockés en base binaire. Quant-à-eux, les nombres réels peuvent être stockés sous plusieurs formes.

Un format peut être évalué selon plusieurs critères:
\begin{itemize}
    \item Nombre de bits utilisés
    \item Précision de la représentation d'un nombre
    \item Gamme dynamique (rapport entre les valeurs maximales et minimales qui peuvent être représentées avec ce format)
    \item Précision des calculs
    \item Complexité de l'implémentation hardware
\end{itemize}

Quelques formats sont expliqués dans cette section, cependant d'autres formats existent également.

\subsection{Virgule fixe}

Le principe des nombres à virgule fixe est de stocker les nombres en tant qu'entiers et d'y appliquer un facteur d'ajustement constant. Ainsi, on peut décider que le nombre représenté est égal à l'entier divisé par 10'000. Lorsqu'on veut avoir plus de chiffres après la virgule, on réduit également la valeur maximale que peut stocker ce type. Ce format a ainsi des limites importantes. La virgule fixe est utilisée lorsque le processeur ne possède pas de d'unité de calcul pour la virgule flottante.

\subsection{Virgule flottante}

Le principal problème de la virgule fixe est qu'il est impossible de stocker à la fois un nombre très grand et un nombre très petit dans le même format. Il est cependant possible d'utiliser les bits à disposition de façon plus intelligente. Les nombres réels peuvent aussi être stockées sous la forme d'une notation scientifique tel que: $-2.568 * 10^{-6}$. Ainsi, les nombres peuvent être représentés sous la forme suivante: $signe * mantisse * base^{exposant}$. Parmi ces quatre éléments, la base n'est pas stockée. La base est souvent égale à 2.

Les ordinateurs actuels utilisent tous les nombres à virgule flottante et la spécification a été révisée en 2019 \cite{ieee-754-2019}.

\subsection{Unums}

Les \textbf{universal numbers} (\textbf{unums}) sont un nouveau format de stockage dont la première version a été publié en 2015 par M. John L. Gustafson. Durant les années suivantes, les unums ont été modifiés a deux reprises. La liste des versions des unums sont visibles dans l'introduction de l'arithmétique des posits (unums III) \cite{posit-arithmetic}.

Le créateur des unums comparent les différents versions des unums avec la virgule flottante et affirme qu'ils sont meilleurs dans tous les cas dans sa présentation lors d'un séminaire à Stanford \cite{youtube-beyond-floating-point}. D'après lui, les unums offrent, entre autres, :

\begin{itemize}
    \item Une meilleure précision
    \item Une gamme dynamique plus large
    \item Meilleures réponses avec le même nombre de bits
    \item Réduction de la consommation d'énergie et de la latence
\end{itemize}

Beaucoup plus d'informations sur les unums sont également disponibles sur le site \url{posithub.org} \cite{posithub}.

