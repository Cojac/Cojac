\section{Implémentation}

Cette section détaille les points importants nécessaires à l'implémentation des \glspl{Posit} dans \gls{COJAC}.

\subsection{Passerelle JNI}

\begin{minipage2}
Tout d'abord, la classe \textit{Posit32Utils} a été écrite avec tous les appels de méthodes natifs.

\begin{minted}{Java}
/* ====================
 * Arithmetic methods
 * ====================
 *
 * All the methods below take one or multiple posit32 as inputs.
 *
 * Use #toPosit to convert a float in a posit.
 */
public native static float add(float a, float b);

public native static float substract(float a, float b);
\end{minted}
\end{minipage2}

\begin{minipage2}
Ensuite, il faut générer le header correspondant avec la commande suivante:

\begin{minted}[breaklines]{Shell}
src\main\java> javac -h ..\resources\native-libraries\posits\src\include com\github\cojac\utils\Posit32Utils.java
\end{minted}

Il faut garder le fichier généré tel quel. Le nom des méthodes est important pour pouvoir réaliser la passerelle entre Java et le code natif.
\end{minipage2}

Dans le fichier source qui contient l'implémentation des méthodes déclarées dans le fichier header généré précédemment. Une ligne a été ajoutée pour redéfinir un type qui n'est pas connu de tous les compilateurs:

\begin{minted}{C}
#define __int64 long long

#include "com_github_cojac_utils_Posit32Utils.h"
\end{minted}

Deux macros et une union ont également été écrits afin de pouvoir changer le type des données:

\begin{minted}{C}
#define F2P(VALUE) (((float_posit32_t) {.jfloat = (VALUE)}).posit)
#define P2F(VALUE) (((float_posit32_t) {.posit = (VALUE)}).jfloat)

typedef union float_posit32 {
    jfloat jfloat;
    posit32_t posit;
} float_posit32_t;
\end{minted}

Voici un exemple qui montre l'implémentation d'une méthode et son utilisation de ces macros.

\begin{minted}{C}
JNIEXPORT jfloat JNICALL Java_com_github_cojac_utils_Posit32Utils_add
        (JNIEnv *env, jclass positClass, jfloat a, jfloat b) {
    return P2F(p32_add(F2P(a), F2P(b)));
}
\end{minted}

\subsection{Build de la librairie}

\begin{minipage2}
Des fichiers de configurations sont disponibles pour pouvoir créer la librairie. La librairie \gls{SoftPosit} possède déjà un \gls{Makefile} même si des modifications ont dû être effectuées comme mentionné dans les sections \ref{sec:problem_softposit_compilation} et \ref{sec:problem_softposit_include}. Un CMakeLists.txt a été créé pour la compilation de la librairie qui fait la liaison avec \gls{JNI}.

\inputmintedcolor{CMake}{code/CMakeLists.txt}
\end{minipage2}

\begin{minipage2}
Les lignes importantes sont décrites ci-dessous:

\begin{itemize}
    \item Lignes 5-7: Cmake peut ne pas réussir à trouver certains éléments de Java et \gls{JNI}, qui ne sont pas nécessaires pour ce projet. Ainsi, ces lignes indiquent qu'il n'a pas besoin de les chercher.
    \item Lignes 9-10: Il faut trouver Java, puis \gls{JNI} afin de pouvoir inclure les headers nécessaires pour la compilation de la librairie.
    \item Lignes 13 et 24: Ces lignes indiquent les dossiers qui contiennent les headers nécessaire à \gls{JNI}.
    \item Ligne 19: Cette ligne spécifie qu'une librairie doit être créée.
    \item Ligne 26: Cette ligne permet d'ajouter la librairie \gls{SoftPosit} à l'intérieur de la librairie générée.
\end{itemize}
\end{minipage2}

\begin{minipage2}
Tout d'abord, il faut compiler la librairie \gls{SoftPosit} avec les commandes suivantes:
\begin{minted}{Shell}
cd libraries/SoftPosit/build/Linux-x86_64-GCC
make
cd ../../../..
\end{minted}
\end{minipage2}

Ensuite, les commande suivantes permettent de compiler la librairie qui sert de passerelle avec \gls{JNI}:
\begin{minted}{Shell}
cd ..
cmake src
make
\end{minted}

Windows 10 avec mingw64 a été utilisé pour compiler la librairie pour Windows 64 bits (.dll). L'application Ubuntu 20.04 sur Windows 10 a été utilisée avec le compilateur gcc pour compiler pour Linux 64 bits (.so).

\subsection{Chargement de la librairie}

Le chargement de la librairie est effectuée en se basant sur la classe \textit{ConversionBehaviour} qui doit déjà charger une librairie native. Ainsi, les librairies natives sont compilées et ajoutées dans le \gls{JAR}. Ensuite, une méthode est créée pour charger la librairie correspondant à la plateforme sur laquelle le \gls{JAR} est exécuté:

\begin{minted}[breaklines]{Java}
public static void loadLibrary() {
    String libRoot = "/native-libraries/posits/";
    String winLib64 = libRoot + "posits_jni.dll";
    String linLib64 = libRoot + "libposits_jni.so";
    String OSName = System.getProperty("os.name");
    int arch = Integer.parseInt(System.getProperty("sun.arch.data.model"));
    try {
        if (OSName.startsWith("Windows")) {
            if (arch == 64) {
                NativeUtils.loadLibraryFromJar(winLib64);
            }
        } else if (OSName.startsWith("Linux")) {
            if (arch == 64) {
                NativeUtils.loadLibraryFromJar(linLib64);
            }
        } else {
            throw new UnsupportedOperationException("This operation is not supported on this platform");
        }
    } catch (IOException e) {
        throw new RuntimeException(e);
    }
}
\end{minted}

Cette méthode est chargée lorsque l'option indiquant l'utilisation des \glspl{Posit} est traitée afin de charger la librairie uniquement si elle est nécessaire.

\subsection{Autres}

Un nouveau \gls{Wrapper} a été implémenté pour les \glspl{Posit} 32 de manière similaire aux \glspl{Complex-number}. Cette étape est décrite dans la section \ref{sec:complex_implementation_wrapper}.

Une nouvelle méthode la librairie standard, \textit{Math.fma()}, est désormais aussi remplacée par une méthode du \gls{Wrapper}. Le processus utilisé est identique à celui détaillé dans la section \ref{sec:complex_implementation_methods} sur les \glspl{Complex-number}.

