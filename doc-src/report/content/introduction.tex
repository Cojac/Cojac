\chapter{Introduction}

La plupart des langages de programmation offrent des capacités similaires pour stocker des nombres et effectuer des calculs. Ces langages permettent, entre autres, d'utiliser  des nombres réels.

Ces nombres réels ont tout de même des limitations. Ils sont limités au domaine du réel. De plus, les nombres à virgule flottante sont souvent inexacts. Par conséquent, lorsque les erreurs s'accumulent, le résultat d'un calcul peut être très éloigné de la réponse exacte.

Pourtant, d'autres alternatives existent. Les \glspl{Complex-number} sont couramment utilisés en mathématique et en physique. Ils peuvent être utilisés pour simplifier des réponses utilisant des racines de nombres négatifs ou pour combiner deux quantités réelles telles que la tension et l'intensité en électricité. Quant à eux, les nombres réels peuvent aussi être représentés différemment. Les \textbf{universal numbers} (\textbf{\glspl{Unums}}) sont une représentation alternative possible dont la dernière version est expliquée dans l'article \textit{Beating Floating Point at its Own Game: Posit Arithmetic} \cite{posit}.


\section{Contexte}

En Java, aucun type ni aucune classe dans le \gls{JDK} permet de gérer des \glspl{Complex-number} ou des \glspl{Unums}, mais il est possible de l'implémenter soi-même. Des librairies pouvant gérer ces éléments peuvent déjà exister.

La première solution pour ajouter ces fonctionnalités et de créer de nouvelles classes: une classe pour les \glspl{Complex-number} et une classe pour les \glspl{Unums}. Ensuite, il faut écrire le code en utilisant directement ces classes. Pour changer le type de calculs effectué (ex: \gls{Complex-number} $\rightarrow$ nombre réel), il faut changer le code source de l'application.

\gls{COJAC} \cite{COJAC} est une librairie Java permettant de modifier les capacités arithmétiques d'un programme Java sans en modifier le code. Elle utilise l'API d'instrumentation et peut transformer les classes et méthodes au runtime pour changer le type de calcul effectué. Ainsi, pour changer le type de calculs effectué (ex: nombre réel $\rightarrow$ \gls{Complex-number}), il faut seulement changer l'argument donné à \gls{COJAC} lors du démarrage de l'application. Ceci ne demande aucune modification dans l'application de l'utilisateur.

\section{Objectifs}

Le but de ce projet est d'ajouter deux nouvelles fonctionnalités à \gls{COJAC}. \gls{COJAC} devra permettre de remplacer automatiquement les nombres à virgules flottantes par deux nouveaux types numériques.

\subsection{Intégration des nombres complexes}

Une option de \gls{COJAC} permettra de changer le comportement des calculs dans l'application de l'utilisateur. Les nombres à virgules flottantes (\textit{float} et \textit{double}) seront remplacés, au runtime, par des \glspl{Complex-number}. Les opérations arithmétiques telles que l'addition, la soustration, etc. devront être adaptées. De plus, les méthodes souvent utilisées de la librairie standard devront également pouvoir fonctionner. Par exemple, la méthode \textit{Math.sqrt} devra permettre de retourner la racine carrée d'un nombre négatif.

Voici un exemple sans les \glspl{Complex-number}:
\begin{minted}{Java}
double val = Math.sqrt(-1); // = NaN
val = val * val; // = NaN;
\end{minted}

Avec les \glspl{Complex-number}, on obtient le résultat suivant:
\begin{minted}{Java}
double val = Math.sqrt(-1); // = i
val = val * val; // = -1;
\end{minted}

\subsection{Intégration des Unums}

Une option de \gls{COJAC} permettra de changer le format de stockage et de calculs des nombres réels. Les \glspl{Unums} seront utilisés à la place de la virgule flottante. Par conséquent, les opérations arithmétiques devront être redéfinies pour fonctionner avec ce nouveau format de stockage. Il faudra probablement utiliser \gls{JNI} pour accéder à une librairie C/C++ permettant d'utiliser les \glspl{Unums}, mais d'autres approches restent possibles.

\subsection{Démonstration des deux fonctionnalités}

Des programmes de démonstrations seront réalisés pour montrer ces deux fonctionnalités. Ces démonstrations doivent montrer l'utilité et les avantages de cette approche.

\begin{minipage2}
\section{Objectifs secondaires}

D'autres ajouts de fonctionnalités ou modifications permettraient d'améliorer ce projet.

\subsection{Mise à jour des librairies}

\gls{COJAC} utilise plusieurs librairies dont les versions sont désormais obsolètes. Il vaut mieux mettre à jour les versions avant de rencontrer des problèmes à cause de versions trop anciennes. Cependant, \gls{COJAC} devra garantir une compatibilité pour Java 8+.
\end{minipage2}

\subsection{Tests de performance}

Lorsque les fonctionnalités de remplacement des nombres à virgule flottante par des \glspl{Complex-number} et des \glspl{Unums}, il restera encore un aspect inconnu qui est pourtant important pour décider de l'utilité de cette fonctionnalité: les performances. Pour cette raison, des tests de performance peuvent aussi être ajoutés pour tester l'efficacité de l'implémentation.

\subsection{Documentation et promotion}

\gls{COJAC} possède une documentation pour l'utiliser et des vidéos pour expliquer l'utilité de certaines fonctionnalités. Il serait possible de documenter les nouvelles fonctionnalités ajoutées, de réaliser une vidéo pour montrer l'utilité de ces vidéos ou encore de compléter la documentation actuelle.

\subsection{Comparaison des approches Wrappers et Behaviours}

Deux approches sont possibles pour implémenter de nouvelles fonctionnalités dans \gls{COJAC}:

\begin{itemize}
    \item \Gls{Behaviour}: les opérations sur les floats et les doubles sont simplement remplacées par un appel de méthode. Ainsi, il est possible de changer le comportement de ceux-ci. Dans ce projet, il serait possible de mettre la partie réelle et la partie imaginaire dans un double et de modifier les opérations qui les utilisent.
    \item \Gls{Wrapper}: les floats et les doubles peuvent être remplacés par un objet (\gls{Wrapper}). Ce qui permet d'ajouter plus d'éléments dans le \gls{Wrapper}.
\end{itemize}

\subsection{Améliorations diverses}

Toute autre amélioration à la base de code existante est aussi le bienvenu. Voici quelques exemples d'améliorations qui pourraient être effectuées:

\begin{itemize}
    \item Ajout d'un CI sur GitLab pour vérifier les tests et compiler le \gls{JAR}.
    \item Ajout d'un logger pour améliorer la gestion des logs.
    \item Améliorer l'architecture du projet.
    \item Améliorer la documentation du code.
    \item Faire les TODO présents dans le code.
\end{itemize}

\section{Structure du rapport}

Ce rapport décrit le déroulement de ce projet ainsi que les problèmes, les décisions et les évolution de ce dernier. Les différentes parties de ce rapport sont décrites ci-dessous:

\begin{itemize}
    \item Introduction: Ce chapitre explique le contexte et définit les objectifs de ce projet.
    \item \Gls{COJAC}: Ce chapitre décrit les notions nécessaires à la compréhension du fonctionnement de \gls{COJAC}.
    \item Intégration des \glspl{Complex-number}: Ce chapitre explique l'intégration complète des nombres complexes. Il décrit les  \glspl{Complex-number}, les décisions prises, l'implémentation, les tests, etc.
    \item Intégration des \glspl{Unums}: Ce chapitre explique l'intégration complète des \glspl{Unums}. Il décrit les \glspl{Unums}, les choix effectués pour l'implémenter, des tests, etc.
    \item Versions: Ce chapitre liste la version des librairies utilisées.
    \item Conclusion: Ce chapitre résume l'état du projet par rapport à ses objectifs ainsi que ses possibilités d'amélioration.
    \item Références: Ce chapitre liste les sources sur lesquels se basent ce projet.
    \item Glossaire: Définit certains termes utilisés dans le rapport.
\end{itemize}