\section{Implémentation}

\subsection{Wrapper}

La création d'un nouveau wrapper se fait en créant une nouvelle classe héritant de \textit{ACojacWrapper}. Comme \textit{ACojacWrapper} est une classe abstraite, le wrapper ne peut pas étendre d'autres classes.

Il faut également implémenter un constructeur prenant un \textit{ACojacWrapper} en paramètre. Il faut également gérer le cas où ce paramètre est null. Voici le constructeur implémenté pour ce wrapper:

\begin{minted}{Java}
public WrapperComplexNumber(ACojacWrapper w) {
    // CommonDouble can call this constructor with a null wrapper
    if (w == null) {
        this.complex = new Complex(0, 0);
    } else {
        Complex value = ((WrapperComplexNumber) w).complex;
        this.complex = new Complex(value.getReal(), value.getImaginary());
    }
}
\end{minted}

\subsection{Remplacement de méthodes}

Seulement une sélection de méthodes de la librairie standards sont remplacées. Deux méthodes supplémentaires sont également remplacées car elles sont utilisées dans la démonstration. Il s'agit de:
\begin{itemize}
    \item \textit{Math.cbrt(double)}
    \item \textit{Double.isNaN(double)}
\end{itemize}

Le remplacement des méthodes de la librairie standard pour les wrappers se fait dans la classe \textit{com.github.cojac.instrumenters.ReplaceFloatsMethods} dans la méthode \textit{fillMethods}. Les lignes suivantes ont été ajoutées pour remplacer ces deux méthodes supplémentaires:

\begin{minted}[breaklines]{Java}
// String CDW; // signature of wrapper
// String CDW_N; // name of wrapper
invocations.put(new MethodSignature(DL_NAME, "isNaN", "(D)Z"),
new InvokableMethod(CDW_N, "double_isNaN", "(" + CDW + ")Z", INVOKESTATIC));

invocations.put(new MethodSignature(MATH_NAME, "cbrt", "(D)D"),
new InvokableMethod(CDW_N, "math_cbrt", "(" + CDW + ")" + CDW, INVOKESTATIC));
\end{minted}

Il faut également ajouter les deux méthodes déclarées ici: \textit{double\_isNaN} and \textit{math\_cbrt} dans la classe \textit{com.github.cojac.models.wrappers.CommonDouble} et les faire appeler la méthode correspondante du wrapper:

\begin{minted}{Java}
public static boolean double_isNaN(CommonDouble a){
    return a.val.isNaN();
}

public static CommonDouble math_cbrt(CommonDouble a){
    return new CommonDouble(a.val.math_cbrt());
}
\end{minted}

\subsection{Ajout de l'option}

Une nouvelle option doit aussi être ajoutée à COJAC pour que le wrapper puisse être utilisé et configuré. Tout d'abord, il faut ajouter une valeur dans l'enum \textit{com.github.cojac.Arg}.

\begin{minted}{Java}
COMPLEX_NUMBER ("Rc"),
\end{minted}

Cette enum a aussi une méthode \textit{createOptions}. Il faut aussi y ajouter une option. Ceci permettra de rendre l'option publique et utilisable comme argument et permettra aussi d'afficher l'aide à propos de cette option. Voici le code pour ajouter une option avec un argument facultatif:

\begin{minted}[breaklines]{Java}
options.addOption(OptionBuilder
    .withArgName("strict")
    .hasOptionalArg()
    .withDescription("Use complex number wrapping. Strict mode generates an exception when the imaginary " +
            "part is lost or when a comparison between two different complex numbers is made.")
    .create(COMPLEX_NUMBER.shortOpt()));
\end{minted}

Il faut aussi détecter quand l'option est utilisée et la traiter en conséquence. Pour ce faire, il faut ajouter une condition dans la classe \textit{com.github.cojac.CojacReferences.CojacReferencesBuilder} à l'intérieur de la méthode \textit{build}. Ce code doit être ajouté suffisamment haut dans la méthode car elle définit l'option \textit{Arg.NG\_WRAPPER} qui est aussi traitée dans la même méthode. Voici l'extrait de code qui permet de traiter cette option:

\begin{minted}[breaklines]{Java}
if (args.isSpecified(Arg.COMPLEX_NUMBER)) {
    args.setValue(Arg.NG_WRAPPER, "com.github.cojac.models.wrappers.WrapperComplexNumber");
    WrapperComplexNumber.setStrictMode( args.getValue(Arg.COMPLEX_NUMBER) != null);
}
\end{minted}

\subsection{\textit{toString} et \textit{fromString}}

La méthode \textit{toString} a été implémentée pour avoir trois formats possibles d'après le type de nombres complexes.

\begin{itemize}
    \item Nombre réel: uniquement la partie réelle (ex: 2.25).
    \item Nombre imaginaire: uniquement la partie réelle avec un \textit{i} à la fin (ex: -4.2i).
    \item Nombre complexe: les deux parties sont affichées (ex: -4.25 + 4.0i ou 1.25 - 1.5i).
\end{itemize}

Ces deux méthodes se basent sur l'affichage du double de Java. Ainsi, \textit{1.401298464324817E-45 - 1.0i} est une représentation possible d'un nombre complexe et il est aussi possible de convertir cette chaîne de caractères en un nombre complexe.

La méthode \textit{fromString} est capable de lire et de créer un nombre complexe à partir de tous les exemples de String valides donnés précédemment.

Cependant, des modifications ont dû être effectuées parce que les méthodes \textit{Float.parseFloat} et \textit{Double.parseDouble} appellent les méthodes des classes \textit{CommonFloat} et \textit{CommonDouble}. Cependant, ces classes font ensuite elles-même un appel aux vraies méthodes \textit{Float.parseFloat} et \textit{Double.parseDouble}.

Voici l'ancien code du \textit{CommonDouble}:

\begin{minted}{Java}
public CommonDouble(String v) {
    this(Double.valueOf(v));
}

public static CommonDouble fromString(String a){
    return fromDouble(Double.valueOf(a));
}
\end{minted}

\begin{minipage2}
Et voici le nouveau code qui fait l'appel à une nouvelle méthode du \textit{ACojacWrapper}:

\begin{minted}{Java}
public CommonDouble(String v) {
    val = newInstance(null).fromString(v, false);
}

public static CommonDouble fromString(String a){
    return new CommonDouble(a);
}
\end{minted}
\end{minipage2}
