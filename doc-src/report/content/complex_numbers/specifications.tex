\section{Spécifications}

L'intégration des \glspl{Complex-number} doit permettre de pouvoir réaliser toutes les opérations arithmétiques standards comme l'addition ou la multiplication. Les comparaisons doivent pouvoir fonctionner avec les nombres réels. Il y aura deux méthodes magiques pour obtenir la partie réelle et imaginaire des \glspl{Complex-number}. Lorsqu'un \gls{Complex-number} sera converti en string, il sera affichée sous sa forme algébrique. Les opérations communes de la librairie standard telles que \textit{Math.sqrt}, \textit{Math.sin}, etc. devront aussi être adaptées.

\section{Démonstration}

Le code suivant sera utilisé pour montrer l'intérêt des \glspl{Complex-number}. Il permettra aussi de vérifier que l'implémentation des \glspl{Complex-number} est correcte. Le code suivant permet de trouver une racine d'un polynôme du 3e degré. Les 2 polynômes ont des racines réelles et pourtant, le polynôme $x^3 - 2x^2 - 13x - 10$ produit un \textit{NaN} (not a number) lors du calcul de la racine. Si le calcul est effectué avec des \glspl{Complex-number}, une des racines devrait être trouvée.
\begin{minted}[breaklines]{Java}
// Find a root of a cubic equation of the form ax^3 + bx^2 + cx + d = 0 with the general cubic formula
// This formula can be found on wikipedia: https://en.wikipedia.org/wiki/Cubic_equation#General_cubic_formula
static double findRootOfCubicEquation(double a, double b, double c, double d) {
    double det0 = b * b - 3 * a * c;
    double det1 = 2 * b * b * b - 9 * a * b * c + 27 * a * a * d;

    double sqrt = Math.sqrt(det1 * det1 - 4 * det0 * det0 * det0);
    // the root can't be calculated if this value is not available.
    if (Double.isNaN(sqrt)) return Double.NaN;

    double coef = Math.cbrt((det1 + sqrt) / 2);

    if (coef == 0) coef = Math.cbrt((det1 - sqrt) / 2);
    if (coef == 0) return -b / (3 * a);
    return -(b + coef + det0 / coef) / (3 * a);
}

public static void main(String[] args) {
    System.out.println(findRootOfCubicEquation(2, 1, 3, 1) + " should be ~0.66666...");
    System.out.println(findRootOfCubicEquation(1, -2, -13, -10) + " should be -1, -2 or 5");
}
\end{minted}