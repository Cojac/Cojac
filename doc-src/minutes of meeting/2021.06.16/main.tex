%%
%% This is file `./samples/minutes.tex',
%% generated with the docstrip utility.
%%
%% The original source files were:
%%
%% meetingmins.dtx  (with options: `minutes')
%% ----------------------------------------------------------------------
%% 
%% meetingmins - A LaTeX class for formatting minutes of meetings
%% 
%% Copyright (C) 2011-2013 by Brian D. Beitzel <brian@beitzel.com>
%% 
%% This work may be distributed and/or modified under the
%% conditions of the LaTeX Project Public License (LPPL), either
%% version 1.3c of this license or (at your option) any later
%% version.  The latest version of this license is in the file:
%% 
%% http://www.latex-project.org/lppl.txt
%% 
%% Users may freely modify these files without permission, as long as the
%% copyright line and this statement are maintained intact.
%% 
%% ----------------------------------------------------------------------
%% 
\documentclass[11pt]{meetingmins}


\usepackage{hyperref}
\usepackage{tabularx}

% To include images
\usepackage{graphicx}

% text color
\usepackage{xcolor}


%% CONFIG %%
% Default image directory
\graphicspath{{images/}}


\setcommittee{Intégration des nombres complexes et des Unums en Java avec COJAC}

\setmembers{
    Frédéric Bapst (Superviseur),
    Cédric Tâche (Etudiant)
}

\setpresent{
    Frédéric Bapst (Superviseur),
    Cédric Tâche (Etudiant)
}

\setlength{\headheight}{13.6pt}

\date{16 juin 2021}

\begin{document}

\begin {center} {
    \large \textbf {Intégration des nombres complexes et des Unums en Java avec COJAC}
}
\vspace {0.5ex}

PV de la séance du 16 juin 2020 (9h30 - 10h20) via Teams
\end {center} \vspace {1.5em}

\noindent
\textbf{Présents:} Frédéric Bapst, Cédric Tâche

\section{Ordre du jour}
Les points suivants ont été abordés durant la séance:
\begin{hiddenitems}
    \item Validation du PV du 9 juin 2021.
    \item Cahier des charges v1.2
    \item Rapport v0.3
    \item Démonstration
    \item Planification
\end{hiddenitems}

\section{PV}
\begin{hiddenitems}
    \item Le PV du 9 juin a été validé.
\end{hiddenitems}

\section{Général}
\begin{hiddenitems}
    \item Pour que M. Bapst et M. Wicht puissent avoir accès aux références, les PDF peuvent être ajoutés sur le dépôt Git dans un dossier "références".
    \item Les derniers tests ajoutés au projet étaient le NumericalProfilerTest. Ils peuvent servir de modèles pour tester les nombres complexes et les unums.
    \item Le fix de M. Bapst provoquait une erreur dans les tests. La branche sera push sur le dépôt afin qu'il puisse regarder les problèmes.
\end{hiddenitems}

\section{Conception des nombres complexes}
\begin{hiddenitems}
    \item Il faut bien réfléchir à la comparaison des nombres complexes et justifier les choix.
    \item La comparaison entre les nombres réels doit toujours être correct.
    \item Un warning ou une exception peut être généré si la comparaison ne fait pas de sens (ex: comparaison entre deux nombres complexes).
    \item Il faut aussi gérer les NaN qui existent dans les doubles.
    \item Un Wrapper sera utilisé afin de garder la précision des doubles.
    \item Des librairies externes pour utiliser les nombres complexes seront cherchées et comparées.
    \item M. Bapst sera averti lorsque le choix entre une librairie externe et une nouvelle implémentation pour les nombres complexes sera décidé.
\end{hiddenitems}

\section{Démonstration}
\begin{hiddenitems}
    \item La démonstration permet de trouver une solution d'un polynôme du 3e degré. Cependant, le calcul nécessite des nombres complexes pour trouver une solution à chaque fois et ce, même si la solution est réelle
    \item Ainsi, elle permet de montrer l'utilité des nombres complexes.
\end{hiddenitems}

\section{Visite d'expert}
\begin{hiddenitems}
    \item La visite d'expert aura lieu le lundi 16 juin.
    \item Même si les anciens rapports ne sont pas publics, ils peuvent être référencés.
    \item Le cahier des charges sera envoyé à M. Wicht avant la visite d'expert
\end{hiddenitems}

\section{Planification}
\begin{hiddenitems}
    \item Le projet a environ 0.5 - 1 jour de retard par rapport à la planification.
\end{hiddenitems}

\section{Décision}
\begin{hiddenitems}
    \item Un Wrapper sera utilisé afin de garder la précision des doubles.
    \item M. Bapst sera averti lorsque le choix entre une librairie externe et une nouvelle implémentation pour les nombres complexes sera décidé.
\end{hiddenitems}

\section{Prochaines tâches}
\begin{hiddenitems}
    \item Finir la conception des nombres complexes.
    \item Implémenter l'intégration des nombres complexes.
    \item Tester l'intégration des nombres complexes.
\end{hiddenitems}

\vspace{1em}
\par \noindent \textbf {Prochaine séance:} Le mercredi 23 juin 2020 à 9h30

\end{document}
%% 
%% Copyright (C) 2011-2013 by Brian D. Beitzel <brian@beitzel.com>
%% 
%% This work may be distributed and/or modified under the
%% conditions of the LaTeX Project Public License (LPPL), either
%% version 1.3c of this license or (at your option) any later
%% version.  The latest version of this license is in the file:
%% 
%% http://www.latex-project.org/lppl.txt
%% 
%% Users may freely modify these files without permission, as long as the
%% copyright line and this statement are maintained intact.
%% 
%% This work is "maintained" (as per LPPL maintenance status) by
%% Brian D. Beitzel.
%% 
%% This work consists of the file  meetingmins.dtx
%% and the derived files           meetingmins.cls,
%%                                 sampleminutes.tex,
%%                                 department.min,
%%                                 README.txt, and
%%                                 meetingmins.pdf.
%% 
%%
%% End of file `./samples/minutes.tex'.
