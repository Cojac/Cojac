%%
%% This is file `./samples/minutes.tex',
%% generated with the docstrip utility.
%%
%% The original source files were:
%%
%% meetingmins.dtx  (with options: `minutes')
%% ----------------------------------------------------------------------
%% 
%% meetingmins - A LaTeX class for formatting minutes of meetings
%% 
%% Copyright (C) 2011-2013 by Brian D. Beitzel <brian@beitzel.com>
%% 
%% This work may be distributed and/or modified under the
%% conditions of the LaTeX Project Public License (LPPL), either
%% version 1.3c of this license or (at your option) any later
%% version.  The latest version of this license is in the file:
%% 
%% http://www.latex-project.org/lppl.txt
%% 
%% Users may freely modify these files without permission, as long as the
%% copyright line and this statement are maintained intact.
%% 
%% ----------------------------------------------------------------------
%% 
\documentclass[11pt]{meetingmins}


\usepackage{hyperref}
\usepackage{tabularx}

\setcommittee{Intégration des nombres complexes et des Unum en Java avec COJAC}

\setmembers{
    Frédéric Bapst (Superviseur),
    Cédric Tâche (Etudiant)
}

\setpresent{
    Frédéric Bapst (Superviseur),
    Cédric Tâche (Etudiant)
}

\setlength{\headheight}{13.6pt}

\date{31 mai 2021}

\begin{document}

\begin {center} {
    \large \textbf {Intégration des nombres complexes et des Unum en Java avec COJAC}
}
\vspace {0.5ex}

PV de la séance du 31 mai 2020 (14h00 - 15h30) via Teams
\end {center} \vspace {1.5em}

\noindent
\textbf{Présents:} Frédéric Bapst, Cédric Tâche

\section{Ordre du jour}
Les points suivants ont été abordés durant la séance
\begin{hiddenitems}
    \item L'administratif
    \item L'expert externe
    \item L'organisation
    \item Les documents
    \item COJAC
    \item Les objectifs
    \item L'implémentation
\end{hiddenitems}

\section{Administratif}
\begin{hiddenitems}
    \item Le \textbf{résumé} du travail de bachelor doit être effectué en \textbf{allemand} et être validé pour pouvoir obtenir la mention bilingue.
    \item Si M. Tâche avait soudainement un empêchement durant les jours où ont lieu les défenses orales (du 6 au 8 septembre), il faut prévenir l'administration le plus tôt possible.
\end{hiddenitems}

\section{Expert externe}
\begin{hiddenitems}
    \item Contrairement aux projets de semestre précédents, un expert externe est aussi présent pour le travail de bachelor.
    \item L'expert de ce projet est M. Baptiste Wicht.
    \item L'étudiant, M. Tâche, doit contacter l'expert durant cette semaine pour fixer un premier rendez-vous.
    \item La première séance devrait avoir lieu vers le milieu du projet (vers la semaine du 14 ou du 21 juin).
    \item Les conversations entre l'expert et M. Tâche seront également envoyées, en copie, à M. Bapst.
    \item Il faut demander à l'expert s'il veut être inclus dans l'équipe Teams.
    \item Les rencontres avec l'expert peuvent se faire en présentiel ou à distance. Elles se feront sans M. Bapst.
    \item Le dépôt Git peut être mis en public pour permettre à l'expert d'y accéder.
\end{hiddenitems}

\section{Organisation}
\begin{hiddenitems}
    \item Tous les documents et le code seront mis sur le dépôt Git.
    \item La séance hebdomadaire aura lieu tous les mercredis matins à 9h30.
    \item D'autres séances peuvent être organisées en cas de problèmes ou de nécessité.
    \item Une invitation pour les séances sera envoyée.
\end{hiddenitems}

\section{Documents}
\begin{hiddenitems}
    \item Le PV doit contenir les faits et les décisions. Il doit donner des détails, mais il ne faut pas reporter les discussions complètes.
    \item Le PV doit contenir la date, l'heure et les participants.
    \item Un PV sera aussi réalisé lors des visites de l'expert.
    \item Suite aux retour du projet de semestre, M. Tâche a demandé des exemples de bons PV. M. Bapst essaiera d'en trouver.
    \item Le modèle utilisé pour le rapport, le PV et les autres documents est libre.
    \item Il n'y a pas besoin d'utiliser Git LFS pour la documentation.
\end{hiddenitems}

\section{COJAC}
\begin{hiddenitems}
    \item COJAC est un outil qui modifie les capacités arithmétiques de JAVA.
    \item Il peut détecter certains comportements extrêmes tels que des overflows, des NaN, ...
    \item Il peut aussi modifier le comportement des calculs. Ceci peut inclure le choix de la précision des nombres à virgule flottante ou des calculs automatiques de dérivés.
    \item Cet outil ne peut pas agir sur les librairies standard Java ou natives. Dans ce cas-là, les wrappers où les types spéciaux créés par COJAC doivent être transformés à nouveau en types primitifs, ce qui peut faire perdre de la précision.
\end{hiddenitems}

\section{Objectifs}
\begin{hiddenitems}
    \item Le projet consiste à ajouter deux nouvelles capacités à COJAC: les nombres complexes et les Unums.
    \item Il faudra ajouter des démonstrations pour montrer les avantages de ces deux fonctionnalités (nombres complexes et unum).
    \item Un seul type d'Unum doit être implémenté. Le choix du type implémenté est libre.
    \item Des tests de performance pourraient aussi être faits pour les deux nouvelles fonctionnalités.
    \item Les librairies du projet peuvent aussi être mises à jour.
    \item Une compatibilité minimale avec Java 8 doit être maintenue.
\end{hiddenitems}

\section{Implémentations}
\begin{hiddenitems}
    \item Il existe peut-être déjà une implémentation des nombres complexes en Java.
    \item L'Unum est un format pour représenter les nombres réels. C'est une alternative aux nombres à virgule flottante.
    \item Il pourrait déjà y avoir une implémentation des Unums en Java, mais la librairie actuelle est en C++.
    \item Il y a 2 possibilités pour implémenter ces fonctionnalités: remplacer les doubles par un wrapper ou modifier le comportement des doubles.
    \item Des librairies locales sont présentes pour faire des tests de performance, mais maintenant, il serait possible d'utiliser JMH. Ces librairies sont présentes dans un sous-projet Maven.
\end{hiddenitems}

\section{Tâches}
\subsection{Responsable: M. Bapst}

\begin{table}[ht]
    \begin{tabularx}{\columnwidth}{ | X | p{8em} |}
        \hline
        \textbf{Description} & \textbf{Date limite} \\
        \hline
        Chercher des bons exemples de PV. & 09.06.2021 \\
        \hline
    \end{tabularx}
\end{table}

\subsection{Responsable: M. Tâche}

\begin{table}[ht]
    \begin{tabularx}{\columnwidth}{ | X | p{8em} |}
        \hline
        \textbf{Description} & \textbf{Date limite} \\
        \hline
        Contacter l'expert et fixer un rendez-vous & 04.06.2021 \\
        Ecrire le cahier des charges & 08.06.2021 \\
        Envoyer l'invitation aux prochaines séances & 01.06.2021 \\
        \hline
    \end{tabularx}
\end{table}

\vspace{1em}
\par \noindent \textbf {Prochaine séance:} Le mercredi 2 juin 2020 à 14:00

\end{document}
%% 
%% Copyright (C) 2011-2013 by Brian D. Beitzel <brian@beitzel.com>
%% 
%% This work may be distributed and/or modified under the
%% conditions of the LaTeX Project Public License (LPPL), either
%% version 1.3c of this license or (at your option) any later
%% version.  The latest version of this license is in the file:
%% 
%% http://www.latex-project.org/lppl.txt
%% 
%% Users may freely modify these files without permission, as long as the
%% copyright line and this statement are maintained intact.
%% 
%% This work is "maintained" (as per LPPL maintenance status) by
%% Brian D. Beitzel.
%% 
%% This work consists of the file  meetingmins.dtx
%% and the derived files           meetingmins.cls,
%%                                 sampleminutes.tex,
%%                                 department.min,
%%                                 README.txt, and
%%                                 meetingmins.pdf.
%% 
%%
%% End of file `./samples/minutes.tex'.
