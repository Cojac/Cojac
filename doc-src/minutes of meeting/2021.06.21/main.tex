%%
%% This is file `./samples/minutes.tex',
%% generated with the docstrip utility.
%%
%% The original source files were:
%%
%% meetingmins.dtx  (with options: `minutes')
%% ----------------------------------------------------------------------
%% 
%% meetingmins - A LaTeX class for formatting minutes of meetings
%% 
%% Copyright (C) 2011-2013 by Brian D. Beitzel <brian@beitzel.com>
%% 
%% This work may be distributed and/or modified under the
%% conditions of the LaTeX Project Public License (LPPL), either
%% version 1.3c of this license or (at your option) any later
%% version.  The latest version of this license is in the file:
%% 
%% http://www.latex-project.org/lppl.txt
%% 
%% Users may freely modify these files without permission, as long as the
%% copyright line and this statement are maintained intact.
%% 
%% ----------------------------------------------------------------------
%% 
\documentclass[11pt]{meetingmins}


\usepackage{hyperref}
\usepackage{tabularx}

% text color
\usepackage{xcolor}


\setcommittee{Intégration des nombres complexes et des Unums en Java avec COJAC}

\setmembers{
    Baptiste Wicht (Expert),
    Cédric Tâche (Etudiant)
}

\setpresent{
    Baptise Wicht (Expert),
    Cédric Tâche (Etudiant)
}

\setlength{\headheight}{13.6pt}

\date{21 juin 2021}

\begin{document}

\begin {center} {
    \large \textbf {Intégration des nombres complexes et des Unums en Java avec COJAC}
}
\vspace {0.5ex}

PV de la séance du 21 juin 2020 (17h00 - 17h35) via Teams
\end {center} \vspace {1.5em}

\noindent
\textbf{Présents:} Cédric Tâche (Etudiant), Baptise Wicht (Expert)

\section{Ordre du jour}
Les points suivants ont été abordés durant la séance:
\begin{hiddenitems}
    \item Cahier des charges v1.2
    \item Gestion de projet
    \item Rapport v0.4
    \item Démonstration
\end{hiddenitems}

\section{COJAC}
\begin{hiddenitems}
    \item COJAC permet de détecter des comportements arithmétiques limites (overflow, ...)
    \item COJAC permet aussi d'ajouter de nouvelles capacités aux programmes cibles (calcul par intervalle, précision arbitraire, ...)
    \item M. Wicht a déjà travaillé sur une des premières versions de COJAC et a aussi accompagné un projet plus récent.
\end{hiddenitems}

\section{Objectifs}
\begin{hiddenitems}
    \item L'objectif principal est l'intégration des nombres complexes et des unums avec COJAC. Il faudra aussi créer des démonstrations pour montrer l'utilité de ces fonctionnalités.
    \item Comme COJAC est un projet ancien et que de nombreuses personnes y ont travaillées, il y a beaucoup d'aspects qui peuvent aussi être améliorés: documentation, architecture, logs, CI, version des librairies, ...
\end{hiddenitems}

\section{Gestion de projet}
\begin{hiddenitems}
    \item Il y a une réunion par semaine entre M. Bapst et M. Tâche.
    \item Les communications sont essentiellement réalisés par Teams et parfois par email.
    \item Il y a un dépôt GitLab qui contient tout le code et la documentation (PV compris).
    \item Seul M. Tâche travaille sur ce dépôt GitLab.
    \item Une branche \textit{dev} existe et est majoritairement utilisée. Elle sera ajoutée au \textit{master} lorsque la partie des nombres complexes sera terminée.
    \item Un CI qui créer le JAR de Maven (et qui exécutent les tests) a été ajouté sur le GitLab.
\end{hiddenitems}

\section{Démonstration}
\begin{hiddenitems}
    \item La démonstration permet de montrer l'utilité des nombres complexes conformément aux spécifications.
    \item Une deuxième démonstration est en cours de réalisation pour montrer les méthodes magiques.
\end{hiddenitems}

\section{Etat du projet}
\begin{hiddenitems}
    \item L'implémentation est presque terminée.
    \item Des tests unitaires ont déjà été réalisés.
    \item La démonstration fonctionne comme prévu.
\end{hiddenitems}

\section{Problèmes}
\begin{hiddenitems}
    \item La méthode \textit{Double.isNaN} appelle la méthode du wrapper, mais provoque un cast en double également. Ce qui bloque le fonctionnement d'un mode de comparaison stricte.
\end{hiddenitems}

\section{Améliorations possibles}
\begin{hiddenitems}
    \item Les avantages et inconvénients des wrappers et behaviours peuvent être détaillés conformément à un des objectifs secondaires.
    \item Dire explicitement que la librairie analysée a été utilisée pour gérer les nombres complexes.
    \item Les références peuvent être citées que la première fois.
    \item Lorsqu'on référence les figures, on l'écrit normalement avec une majuscule.
    \item Le cahier des charges n'a pas besoin d'être en annexe. Il faut le voir avec M. Bapst.
    \item Les tests de performance pourraient aussi être exécutés automatiquement par le CI (objectif secondaire).
    \item Pour les unums, au lieu d'utiliser JNI, il pourrait être avantageux d'utiliser JNA.
\end{hiddenitems}


\section{Planification}
\begin{hiddenitems}
    \item Le projet est en accord avec la planification.
    \item Le problème au début pour la compilation de COJAC a retardé le projet d'environ 1 à 2 jours.
    \item Deux objectifs secondaires ont déjà été planifiés. La mise à jour des librairies est très probable. Les tests de performance sont un bon objectif secondaire, mais un autre objectif pourrait être réalisé à la place.
\end{hiddenitems}

\vspace{1em}
\par \noindent \textbf {Prochaine visite:} Le lundi 12 juillet 2021 à 17h00

\end{document}
%% 
%% Copyright (C) 2011-2013 by Brian D. Beitzel <brian@beitzel.com>
%% 
%% This work may be distributed and/or modified under the
%% conditions of the LaTeX Project Public License (LPPL), either
%% version 1.3c of this license or (at your option) any later
%% version.  The latest version of this license is in the file:
%% 
%% http://www.latex-project.org/lppl.txt
%% 
%% Users may freely modify these files without permission, as long as the
%% copyright line and this statement are maintained intact.
%% 
%% This work is "maintained" (as per LPPL maintenance status) by
%% Brian D. Beitzel.
%% 
%% This work consists of the file  meetingmins.dtx
%% and the derived files           meetingmins.cls,
%%                                 sampleminutes.tex,
%%                                 department.min,
%%                                 README.txt, and
%%                                 meetingmins.pdf.
%% 
%%
%% End of file `./samples/minutes.tex'.
