%%
%% This is file `./samples/minutes.tex',
%% generated with the docstrip utility.
%%
%% The original source files were:
%%
%% meetingmins.dtx  (with options: `minutes')
%% ----------------------------------------------------------------------
%% 
%% meetingmins - A LaTeX class for formatting minutes of meetings
%% 
%% Copyright (C) 2011-2013 by Brian D. Beitzel <brian@beitzel.com>
%% 
%% This work may be distributed and/or modified under the
%% conditions of the LaTeX Project Public License (LPPL), either
%% version 1.3c of this license or (at your option) any later
%% version.  The latest version of this license is in the file:
%% 
%% http://www.latex-project.org/lppl.txt
%% 
%% Users may freely modify these files without permission, as long as the
%% copyright line and this statement are maintained intact.
%% 
%% ----------------------------------------------------------------------
%% 
\documentclass[11pt]{meetingmins}


\usepackage{hyperref}
\usepackage{tabularx}

% To include images
\usepackage{graphicx}

% text color
\usepackage{xcolor}


%% CONFIG %%
% Default image directory
\graphicspath{{images/}}


\setcommittee{Intégration des nombres complexes et des Unums en Java avec COJAC}

\setmembers{
    Frédéric Bapst (Superviseur),
    Cédric Tâche (Etudiant)
}

\setpresent{
    Frédéric Bapst (Superviseur),
    Cédric Tâche (Etudiant)
}

\setlength{\headheight}{13.6pt}

\date{30 juin 2021}

\begin{document}

\begin {center} {
    \large \textbf {Intégration des nombres complexes et des Unums en Java avec COJAC}
}
\vspace {0.5ex}

PV de la séance du 30 juin 2021 (9h30 - 10h20) via Teams
\end {center} \vspace {1.5em}

\noindent
\textbf{Présents:} Frédéric Bapst, Cédric Tâche

\section{Ordre du jour}
Les points suivants ont été abordés durant la séance:
\begin{hiddenitems}
    \item Validation du PV du 23 juin 2021.
    \item Intégration des nombres complexes
    \item Correction de l'instrumentation
    \item Rapport v0.5
    \item Résumé/Flyer
    \item Planification
\end{hiddenitems}

\section{PV}
\begin{hiddenitems}
    \item Le PV du 23 juin a été validé
\end{hiddenitems}

\section{Intégration des nombres complexes}
\begin{hiddenitems}
    \item L'implémentation des nombres complexes est terminée.
    \item Les tests unitaires couvrent 100\% du wrapper pour les nombres complexes.
    \item Des tests d'intégration testent l'instrumentation en mode normal et en mode strict.
    \item Le wiki décrit désormais le wrapper pour les nombres complexes.
    \item Le lien dans la table des matières devra être vérifié lorsque le wiki sera ajouté sur GitHub.
    \item La version de COJAC sera 1.6.0 jusqu'à la fin du projet.
\end{hiddenitems}

\section{Correction de l'instrumentation}
\begin{hiddenitems}
    \item Le CI/CD sur GitLab ne fonctionnait plus à cause du test de la méthode \textit{nextUp} dans le test \textit{Double2FloatTest.testDouble2FloatConversion} qui était déjà présent avant depuis longtemps.
    \item Le problème était dû à l'utilisation d'un \textit{break} au lieu d'un \textit{continue} dans la classe \textit{BehaviourInstrumenter} du package \textit{com.github.cojac.instrumenters}.
    \item D'autres \textit{break} suspects seront aussi modifiés.
    \item M. Bapst vérifiera les modifications faites dans la classe \textit{BehaviourInstrumenter} du package \textit{com.github.cojac.instrumenters}. à cause de ses connaissances plus avancées dans cette partie du projet.
\end{hiddenitems}

\section{Rapport v0.5}
\begin{hiddenitems}
    \item L'exemple 1 dans \textbf{Conception} $\rightarrow$ \textbf{Modes} n'est pas très clair.
    \item Il faut aussi vérifier la description de cet exemple.
\end{hiddenitems}

\section{Résumé/Flyer}
\begin{hiddenitems}
    \item Les détails à propos du résumé/flyer devrait être bientôt envoyés par email par l'école.
    \item Cet email devrait aussi contenir les informations à propos du flyer qui doit être écrit en allemand pour avoir la mention bilingue.
\end{hiddenitems}

\section{Planification}
\begin{hiddenitems}
    \item La fin de l'intégration des nombres complexes a pris plus de temps que prévu.
    \item La correction du test unitaire a aussi retardé le projet.
    \item A cause du retard, les librairies ne seront pas mises à jour pour l'instant.
    \item Le projet a actuellement environ 2 jours de retard sur la planification.
\end{hiddenitems}

\section{Unums}
\begin{hiddenitems}
    \item Il n'y a pas besoin de décrire toute la théorie des unums. Ce projet se concentre sur l'intégration des unums dans COJAC.
    \item Des références seront ajoutées pour les lecteurs qui veulent approfondir ce sujet.
    \item Il faudra aussi parler de la compatibilité de cette option sur les différents systèmes d'exploitation si l'implémentation des unums est native.
\end{hiddenitems}

\section{Décision}
\begin{hiddenitems}
    \item M. Bapst vérifiera les modifications faites dans la classe \textit{BehaviourInstrumenter} dans le package \textit{com.github.cojac.instrumenters}.
    \item M. Bapst vérifiera également le wiki lorsqu'il le reportera sur GitHub.
\end{hiddenitems}

\section{Prochaines tâches}
\begin{hiddenitems}
    \item Analyse des unums
    \item Spécification de la démo
    \item Conception des unums
\end{hiddenitems}

\vspace{1em}
\par \noindent \textbf {Prochaine séance:} Le mercredi 7 juillet 2021 à 9h30

\end{document}
%% 
%% Copyright (C) 2011-2013 by Brian D. Beitzel <brian@beitzel.com>
%% 
%% This work may be distributed and/or modified under the
%% conditions of the LaTeX Project Public License (LPPL), either
%% version 1.3c of this license or (at your option) any later
%% version.  The latest version of this license is in the file:
%% 
%% http://www.latex-project.org/lppl.txt
%% 
%% Users may freely modify these files without permission, as long as the
%% copyright line and this statement are maintained intact.
%% 
%% This work is "maintained" (as per LPPL maintenance status) by
%% Brian D. Beitzel.
%% 
%% This work consists of the file  meetingmins.dtx
%% and the derived files           meetingmins.cls,
%%                                 sampleminutes.tex,
%%                                 department.min,
%%                                 README.txt, and
%%                                 meetingmins.pdf.
%% 
%%
%% End of file `./samples/minutes.tex'.
